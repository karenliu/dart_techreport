\section{Dynamic Features}
Physics simulation is the core of DART. The earlier versions of DART
adopted the physics engine, RTQL8 \cite{}, which features forward
simulation based on Lagrangian dynamics
and generalized coordinates. Later, we reimplemented the Featherstone
algorithms using Lie group representation to largely improve the
computation efficiency. The constraint solver, which handles contact,
joint limits, and other Cartesian constraints, has also been optimized
in the later versions of DART. 

\subsection{Lagrangian dynamics and generalized coordinates}
DART is distinguished by its accuracy and stability due to its use of
generalized coordinates to represent articulated rigid body systems,
Featherstone’s algorithm to compute the dynamics of
motion, and Lie group for compact representation and derivation of
dynamic equations.

Articulated human motions can be described by a set of dynamic
equations of motion of multibody systems. Since the direct application
of Newton’s second law becomes difficult when a complex articulated
rigid body system is considered, we use Lagrange’s equations derived
from D'Alembert’s principle to describe the dynamics of motion: 
\begin{equation}
M(\vc{q})\ddot{\vc{q}} + C(\vc{q}, \dot{\vc{q}}) = \vc{Q},
\end{equation}
where $\vc{q}$ is the generalized coordinates that indicate the 
configuration of the skeleton. $M(\vc{q})$ is the mass matrix,
$C(\vc{q},\dot{\vc{q}})$ is the Coriolis and centrifugal term of the
equation of motion, and $\vc{Q}$ is the vector of generalized forces
for all the degrees of freedom in the system.

Once we know how to compute the mass matrix, Coriolis and centrifugal
terms, and generalized forces, we can compute the acceleration in
generalized coordinates, $\ddot{\vc{q}}$ for forward dynamics. Conversely, if we
are given $\ddot{\vc{q}}$ from a motion sequence, we can use these equations of
motion to derive generalized forces for inverse dynamics. However,
directly evaluating the mass matrix and Coriolis term is 
computationally expensive, especially for applications that require
real-time simulation. DART implements Featherstone's articulated rigid
body algorithms to compute forward and inverse dynamics. In addition,
DART uses Lie group representation to further improve efficiency.

\paragraph{Featherstone's algorithms.} In particular, we implement the
recursive Newton-Euler algoritm (RNEA) for inverse dynamics and the
articulated-body algorithm (ABA) for forward dynamics. These two
algorithms are known for their $O(n)$ time with $n$ being the number of
body nodes in the skeleton. 

\paragraph{Lie group representation}
\red{JS} (Describe the high-level ideas of Lie group representation and its advantages).

\subsection{Soft body simulation}
\red{Karen}
\subsection{Joint dynamics}
\red{Karen}

\subsection{Actuators}
\red{Karen}

\subsection{Constraints}
\red{JS}
\paragraph{Contacts}
\paragraph{Cartesian constraints}
\paragraph{LCP solvers}

\subsection{Performance}
\red{JS}
